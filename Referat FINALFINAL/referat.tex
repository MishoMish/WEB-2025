\documentclass[12pt,a4paper]{article}

% Пакети за български език и кодиране
\usepackage[T2A]{fontenc}
\usepackage[utf8]{inputenc}
\usepackage[bulgarian]{babel}

% Пакети за графики, таблици и форматиране
\usepackage{graphicx}
\usepackage{tikz}
\usetikzlibrary{shapes,arrows,positioning,shadows,mindmap}
\usepackage{caption}
\usepackage{subcaption}
\usepackage{booktabs}
\usepackage{array}
\usepackage{longtable}

% Пакети за код
\usepackage{listings}
\usepackage{xcolor}

% Пакети за хипервръзки
\usepackage{hyperref}

% Математически пакети
\usepackage{amsmath}
\usepackage{amssymb}

% Настройки за margins
\usepackage[left=2.5cm,right=2.5cm,top=2.5cm,bottom=2.5cm]{geometry}

% Настройки за hyperref
\hypersetup{
    colorlinks=true,
    linkcolor=blue,
    filecolor=magenta,      
    urlcolor=cyan,
    pdftitle={Prompt Engineering и Web разработка с ChatGPT},
    pdfauthor={Студент},
    pdfsubject={Уеб технологии - 25 издание},
    pdfkeywords={prompt engineering, ChatGPT, web development, AI}
}

% Настройки за код блокове
\definecolor{codegreen}{rgb}{0,0.6,0}
\definecolor{codegray}{rgb}{0.5,0.5,0.5}
\definecolor{codepurple}{rgb}{0.58,0,0.82}
\definecolor{backcolour}{rgb}{0.95,0.95,0.92}

\renewcommand{\lstlistlistingname}{Списък с примерен програмен код}
\renewcommand{\lstlistingname}{Код}

\lstdefinestyle{mystyle}{
    backgroundcolor=\color{backcolour},   
    commentstyle=\color{codegreen},
    keywordstyle=\color{magenta},
    numberstyle=\tiny\color{codegray},
    stringstyle=\color{codepurple},
    basicstyle=\ttfamily\footnotesize,
    breakatwhitespace=false,         
    breaklines=true,                 
    captionpos=b,                    
    keepspaces=true,                 
    numbers=left,                    
    numbersep=5pt,                  
    showspaces=false,                
    showstringspaces=false,
    showtabs=false,                  
    tabsize=2,
    inputencoding=utf8,
    extendedchars=true,
    literate={а}{{\cyra}}1 {б}{{\cyrb}}1 {в}{{\cyrv}}1 {г}{{\cyrg}}1 {д}{{\cyrd}}1 {е}{{\cyre}}1 {ж}{{\cyrzh}}1 {з}{{\cyrz}}1 {и}{{\cyri}}1 {й}{{\cyrishrt}}1 {к}{{\cyrk}}1 {л}{{\cyrl}}1 {м}{{\cyrm}}1 {н}{{\cyrn}}1 {о}{{\cyro}}1 {п}{{\cyrp}}1 {р}{{\cyrr}}1 {с}{{\cyrs}}1 {т}{{\cyrt}}1 {у}{{\cyru}}1 {ф}{{\cyrf}}1 {х}{{\cyrh}}1 {ц}{{\cyrc}}1 {ч}{{\cyrch}}1 {ш}{{\cyrsh}}1 {щ}{{\cyrshch}}1 {ъ}{{\cyrhrdsgn}}1 {ь}{{\cyrsftsgn}}1 {ю}{{\cyryu}}1 {я}{{\cyrya}}1 {А}{{\cyrA}}1 {Б}{{\cyrB}}1 {В}{{\cyrV}}1 {Г}{{\cyrG}}1 {Д}{{\cyrD}}1 {Е}{{\cyrE}}1 {Ж}{{\cyrZH}}1 {З}{{\cyrZ}}1 {И}{{\cyrI}}1 {Й}{{\cyrISHRT}}1 {К}{{\cyrK}}1 {Л}{{\cyrL}}1 {М}{{\cyrM}}1 {Н}{{\cyrN}}1 {О}{{\cyrO}}1 {П}{{\cyrP}}1 {Р}{{\cyrR}}1 {С}{{\cyrS}}1 {Т}{{\cyrT}}1 {У}{{\cyrU}}1 {Ф}{{\cyrF}}1 {Х}{{\cyrH}}1 {Ц}{{\cyrC}}1 {Ч}{{\cyrCH}}1 {Ш}{{\cyrSH}}1 {Щ}{{\cyrSHCH}}1 {Ъ}{{\cyrHRDSGN}}1 {Ь}{{\cyrSFTSGN}}1 {Ю}{{\cyrYU}}1 {Я}{{\cyrYA}}1
}

\lstset{style=mystyle}

% Заглавна информация
\title{
    \vspace{-2cm}
    \begin{center}
        \Large СОФИЙСКИ УНИВЕРСИТЕТ "СВ. КЛИМЕНТ ОХРИДСКИ"\\
        \large ФАКУЛТЕТ ПО МАТЕМАТИКА И ИНФОРМАТИКА
    \end{center}
    \vspace{2cm}
    \textbf{\Huge Prompt Engineering и Web разработка с ChatGPT}\\
    \vspace{1cm}
    \Large Тема №187\\
    \normalsize Курс: Уеб технологии (25 издание)
}
\author{}
\date{}

\begin{document}

\maketitle
\newpage

\tableofcontents
\newpage

\listoffigures
\newpage

\listoftables
\newpage

\lstlistoflistings
\newpage

\section{Увод}
\label{sec:intro}

В съвременната епоха на изкуствен интелект, която се характеризира с експоненциален растеж на възможностите на големите езикови модели (Large Language Models, LLM), prompt engineering се откроява като ключова компетентност за ефективното взаимодействие с AI системи [1]. От пускането на ChatGPT в края на 2022 година, технологичната индустрия е свидетел на радикална трансформация в начина, по който разработчиците подхождат към софтуерното инженерство и уеб разработката [2].

Prompt engineering представлява изкуството и науката за формулиране на ефективни инструкции към AI модели, за да се получат желаните резултати [3]. В контекста на уеб разработката, тази дисциплина предлага безпрецедентни възможности за ускоряване на разработката, автоматизация на рутинни задачи и генериране на високо-качествен код [4].

Настоящият реферат е разработен като част от курса по Уеб технологии (25 издание) и има за цел да изследва пресечната точка между prompt engineering и съвременната уеб разработка. Документът е структуриран съгласно изискванията за интерактивен, семантичен и стилизиран информационен проект, който може лесно да бъде трансформиран в HTML формат за уеб публикация.

\subsection{Актуалност на темата}
\label{subsec:relevance}

Няколко фактора определят актуалността на prompt engineering в уеб разработката:

\begin{itemize}
    \item \textbf{AI бум}: Според проучване на McKinsey от 2023 г., над 50\% от организациите вече използват AI инструменти в поне една бизнес функция [5].
    \item \textbf{Нужда от скорост}: Съвременните разработвателни цикли изискват все по-бързо доставяне на функционалности, където AI може да ускори разработката с до 40\% [6].
    \item \textbf{Промяна в работните процеси}: Появата на "AI-assisted development" променя фундаментално начина, по който се пише, тества и поддържа код [7].
\end{itemize}

\subsection{Цели на реферата}
\label{subsec:goals}

Този реферат си поставя следните цели:
\begin{enumerate}
    \item Да дефинира prompt engineering и основните му техники
    \item Да изследва приложението на ChatGPT в уеб разработката
    \item Да анализира рисковете и предизвикателствата
    \item Да представи практически примери и казуси
    \item Да разгледа етичните и професионални аспекти
    \item Да направи прогноза за бъдещето на професията
\end{enumerate}

\section{Какво е Prompt Engineering}
\label{sec:what-is-pe}

Prompt engineering е процесът на проектиране и оптимизация на инструкции (промптове) към AI модели, за да се постигнат точни, релевантни и полезни резултати [8]. Терминът произлиза от естествено-езиковото взаимодействие с LLM модели като GPT-4, Claude, LLaMA и други [9].

\subsection{Дефиниция и основни концепции}
\label{subsec:definition}

Според OpenAI, prompt engineering включва "систематичното създаване на ефективни промпт формулировки, които максимизират качеството и релевантността на отговорите на AI моделите" [10]. Това включва:

\begin{itemize}
    \item \textbf{Ясност}: Прецизно формулиране на задачата
    \item \textbf{Контекст}: Предоставяне на необходима информация
    \item \textbf{Формат}: Спецификация на желания изход
    \item \textbf{Ограничения}: Дефиниране на рамки и правила
\end{itemize}

\subsection{Роля при взаимодействие с LLM}
\label{subsec:llm-interaction}

LLM моделите функционират на принципа на статистическо предсказване на следващия токен в последователност [11]. Качеството на промпта директно влияе върху:

\begin{enumerate}
    \item \textbf{Точност}: Колко добре моделът разбира задачата
    \item \textbf{Релевантност}: Доколко отговорът отговаря на нуждите
    \item \textbf{Консистентност}: Стабилност при повторни заявки
    \item \textbf{Ефективност}: Време и ресурси за получаване на резултат
\end{enumerate}

Фигура~\ref{fig:prompt-flow} илюстрира основния поток на взаимодействие между потребител, промпт и LLM модел.

\begin{figure}[h]
\centering
\begin{tikzpicture}[node distance=2cm, auto,
    block/.style={rectangle, draw=blue!60, fill=blue!5, very thick, minimum size=10mm, minimum width=30mm, rounded corners},
    line/.style={draw, -latex', very thick}]

    \node [block] (user) {Потребител};
    \node [block, right=of user] (prompt) {Промпт (инструкция)};
    \node [block, right=of prompt] (llm) {LLM Модел (GPT-4)};
    \node [block, below=of llm] (response) {Отговор (текст/код)};

    \path [line] (user) -- (prompt);
    \path [line] (prompt) -- (llm);
    \path [line] (llm) -- (response);
    \path [line] (response) -| (user);
\end{tikzpicture}
\caption{Поток на взаимодействие при prompt engineering}
\label{fig:prompt-flow}
\end{figure}

\section{Основни техники и подходи в Prompt Engineering}
\label{sec:techniques}

Съществуват множество техники за оптимизация на промптове, всяка с различни приложения и предимства [12]. В тази секция ще разгледаме най-важните от тях.

\subsection{Поставяне на роля и контекст}
\label{subsec:role-context}

Една от най-основните техники е задаването на роля и контекст на AI модела [13]. Това се постига чрез инструкции като:

\begin{lstlisting}[caption={Базов prompt за генериране на HTML форма}, label=code:basic-prompt]
You are an experienced web developer specializing in semantic HTML5.
Create a contact form with the following fields:
- Name (required)
- Email (required, validated)
- Message (textarea, required)
- Submit button

Use proper semantic HTML tags and accessibility attributes.
\end{lstlisting}

Тази техника помага на модела да "влезе в ролята" и да генерира по-релевантен отговор [14].

\subsection{Ограничения, формати и изходни структури}
\label{subsec:constraints}

Спецификацията на точен формат и ограничения е критична за получаване на използваем резултат [15]. Виж Код~\ref{code:advanced-prompt} за пример:

\begin{lstlisting}[caption={Advanced prompt с контекст и ограничения}, label=code:advanced-prompt]
Task: Create a responsive navigation menu
Constraints:
- Use semantic HTML5 <nav> element
- Mobile-first approach
- No JavaScript dependencies
- Accessibility: ARIA labels, keyboard navigation
- CSS only (no frameworks)
Output format:
1. HTML structure
2. CSS styles
3. Brief explanation of accessibility features
\end{lstlisting}

\subsection{Итеративни промптове и рефайнмент}
\label{subsec:iterative}

Рядко първият промпт дава перфектен резултат. Итеративният подход включва постепенно уточняване [16]:

\begin{enumerate}
    \item Първоначален промпт и резултат
    \item Анализ на резултата
    \item Идентифициране на проблеми
    \item Уточняване на промпта
    \item Повторение до желан резултат
\end{enumerate}

\subsection{Few-shot промптове}
\label{subsec:few-shot}

Few-shot learning предоставя на модела примери за желания формат и стил [17]. Виж Код~\ref{code:few-shot}:

\begin{lstlisting}[caption={Few-shot prompt с примери}, label=code:few-shot]
Create CSS class names following this pattern:

Example 1:
Component: Button, State: Primary, Size: Large
Class: .btn-primary-lg

Example 2:
Component: Card, State: Featured, Size: Medium
Class: .card-featured-md

Now create for:
Component: Modal, State: Success, Size: Small
\end{lstlisting}

\subsection{Chain-of-thought (CoT)}
\label{subsec:cot}

Chain-of-thought промптването насърчава модела да ``мисли на глас'' и да показва стъпките в разсъжденията си [18]. Това е особено полезно за сложни задачи. Виж Код~\ref{code:cot}:

\begin{lstlisting}[caption={Chain-of-thought prompt за решаване на проблем}, label=code:cot]
Problem: Design a database schema for a blog system.

Think step-by-step:
1. What entities do we need? (users, posts, comments, etc.)
2. What relationships exist between entities?
3. What are the key attributes for each entity?
4. What indexes would optimize queries?
5. Are there any normalization considerations?

Provide your reasoning for each step, then the final schema.
\end{lstlisting}

Важна етична бележка: При използване на CoT, трябва да сме наясно, че моделът симулира разсъждения, но не "мисли" по човешки начин [19]. Резултатите винаги трябва да се валидират.

\subsection{Self-consistency и Toolformer-стил}
\label{subsec:self-consistency}

Self-consistency включва генериране на множество отговори и избор на най-консистентния [20]. Toolformer-стил взаимодействие позволява на AI да "извиква" външни инструменти [21]:

\begin{itemize}
    \item Code linters (ESLint, Prettier)
    \item API validators (Postman, Swagger)
    \item Browser DevTools
    \item Git за version control
\end{itemize}

\subsection{Тестове, проверка и валидация}
\label{subsec:validation}

Критично важна е систематичната проверка на AI генерирания код [22]:

\begin{table}[h]
\centering
\caption{Методи за валидация на AI генериран код}
\label{tab:validation-methods}
\begin{tabular}{@{}lll@{}}
\toprule
\textbf{Метод} & \textbf{Инструменти} & \textbf{Приложение} \\ \midrule
Статичен анализ & ESLint, TSLint & JavaScript/TypeScript \\
Форматиране & Prettier, Black & Код стил \\
Тестване & Jest, Mocha, Pytest & Unit/Integration \\
Accessibility & axe, WAVE & A11y проверки \\
Performance & Lighthouse, WebPageTest & Оптимизация \\
Security & OWASP ZAP, Snyk & Уязвимости \\ \bottomrule
\end{tabular}
\end{table}

Таблица~\ref{tab:validation-methods} показва основните категории валидационни инструменти, които трябва да се използват при работа с AI генериран код.

\section{ChatGPT и Prompt Engineering в Web разработката}
\label{sec:chatgpt-web-dev}

ChatGPT и подобни LLM модели предлагат широк спектър от приложения в уеб разработката, от генериране на boilerplate код до помощ при архитектурни решения [23].

\subsection{Генериране на HTML/CSS/JS шаблони}
\label{subsec:template-generation}

Едно от най-практичните приложения е автоматичното генериране на код шаблони [24]. Виж Код~\ref{code:html-template}:

\begin{lstlisting}[language=HTML, caption={Базов HTML5 шаблон с семантична структура}, label=code:html-template]
<!DOCTYPE html>
<html lang="bg">
<head>
    <meta charset="UTF-8">
    <meta name="viewport" content="width=device-width, initial-scale=1.0">
    <meta name="description" content="Семантична уеб страница">
    <title>Prompt Engineering Demo</title>
    <link rel="stylesheet" href="styles.css">
</head>
<body>
    <header>
        <nav aria-label="Main navigation">
            <ul>
                <li><a href="#home">Начало</a></li>
                <li><a href="#about">За нас</a></li>
                <li><a href="#contact">Контакт</a></li>
            </ul>
        </nav>
    </header>
    
    <main>
        <article>
            <h1>Заглавие на статия</h1>
            <section>
                <h2>Въведение</h2>
                <p>Съдържание...</p>
            </section>
        </article>
    </main>
    
    <footer>
        <p>&copy; 2025 Prompt Engineering Project</p>
    </footer>
    
    <script src="script.js"></script>
</body>
</html>
\end{lstlisting}

\subsection{CSS генериране и layout системи}
\label{subsec:css-generation}

AI моделите могат ефективно да генерират модерни CSS layout-и [25]. Код~\ref{code:css-grid} демонстрира CSS Grid система:

\begin{lstlisting}[language=HTML, caption={CSS Grid layout генериран от AI}, label=code:css-grid]
/* Modern CSS Grid Layout */
.container {
    display: grid;
    grid-template-columns: repeat(auto-fit, minmax(300px, 1fr));
    gap: 2rem;
    padding: 2rem;
    max-width: 1200px;
    margin: 0 auto;
}

.card {
    background: #ffffff;
    border-radius: 8px;
    padding: 1.5rem;
    box-shadow: 0 2px 8px rgba(0, 0, 0, 0.1);
    transition: transform 0.3s ease;
}

.card:hover {
    transform: translateY(-4px);
    box-shadow: 0 4px 16px rgba(0, 0, 0, 0.15);
}

/* Responsive adjustments */
@media (max-width: 768px) {
    .container {
        grid-template-columns: 1fr;
        gap: 1rem;
        padding: 1rem;
    }
}
\end{lstlisting}

\subsection{JavaScript и DOM манипулация}
\label{subsec:js-generation}

ChatGPT може да генерира функционален JavaScript код [26]. Виж Код~\ref{code:js-dom}:

\begin{lstlisting}[language=HTML, caption={JavaScript функция за DOM манипулация}, label=code:js-dom]
/**
 * Dynamically creates and inserts cards into the DOM
 * @param {Array} data - Array of card objects
 * @param {string} containerId - Target container ID
 */
function renderCards(data, containerId) {
    const container = document.getElementById(containerId);
    
    if (!container) {
        console.error(`Container ${containerId} not found`);
        return;
    }
    
    // Clear existing content
    container.innerHTML = '';
    
    // Create and append cards
    data.forEach(item => {
        const card = document.createElement('div');
        card.className = 'card';
        card.setAttribute('role', 'article');
        
        card.innerHTML = `
            <h3>${escapeHtml(item.title)}</h3>
            <p>${escapeHtml(item.description)}</p>
            <a href="${escapeHtml(item.link)}" 
               aria-label="Read more about ${escapeHtml(item.title)}">
                Read more
            </a>
        `;
        
        container.appendChild(card);
    });
}

// Security: HTML escape function
function escapeHtml(text) {
    const map = {
        '&': '&amp;',
        '<': '&lt;',
        '>': '&gt;',
        '"': '&quot;',
        "'": '&#039;'
    };
    return text.replace(/[&<>"']/g, m => map[m]);
}
\end{lstlisting}

\subsection{Архитектура, документация и тестове}
\label{subsec:architecture}

AI може да помогне при проектиране на архитектура и генериране на документация [27]:

\begin{itemize}
    \item \textbf{Architecture diagrams}: Генериране на PlantUML, Mermaid диаграми
    \item \textbf{API документация}: OpenAPI/Swagger спецификации
    \item \textbf{README файлове}: Структурирана проектна документация
    \item \textbf{Unit tests}: Jest, Mocha, Pytest тест казуси
\end{itemize}

\subsection{i18n и локализация}
\label{subsec:i18n}

ChatGPT може да генерира структури за интернационализация [28]. Код~\ref{code:i18n} показва пример:

\begin{lstlisting}[language=HTML, caption={Пример за i18n структура}, label=code:i18n]
// i18n/translations.js
const translations = {
    en: {
        nav: {
            home: 'Home',
            about: 'About',
            contact: 'Contact'
        },
        hero: {
            title: 'Prompt Engineering',
            subtitle: 'AI-Powered Web Development'
        },
        footer: {
            copyright: '(c) 2025 All rights reserved'
        }
    },
    bg: {
        nav: {
            home: 'Начало',
            about: 'За нас',
            contact: 'Контакт'
        },
        hero: {
            title: 'Промпт Инженеринг',
            subtitle: 'Уеб разработка с AI'
        },
        footer: {
            copyright: '(c) 2025 Всички права запазени'
        }
    }
};

// Simple i18n function
function t(lang, key) {
    const keys = key.split('.');
    let value = translations[lang];
    
    for (const k of keys) {
        value = value?.[k];
    }
    
    return value || key;
}

// Usage: t('bg', 'nav.home') => 'Начало'
\end{lstlisting}

\subsection{Добри практики за интеграция в pipeline}
\label{subsec:pipeline}

Интегрирането на AI в разработвателския процес изисква дисциплина [29]:

\begin{enumerate}
    \item \textbf{Планиране}: Дефинирайте какво AI ще генерира
    \item \textbf{Ревю}: Винаги преглеждайте генерирания код
    \item \textbf{Тестване}: Пишете тестове за AI код
    \item \textbf{CI/CD}: Интегрирайте валидация в pipeline
    \item \textbf{Документация}: Отбелязвайте AI-генериран код
\end{enumerate}

\subsection{Рискове при ``blind AI-код''}
\label{subsec:blind-ai-risks}

\begin{figure}[h]
\centering
\begin{tikzpicture}[
    mindmap,
    concept color=red!30,
    every node/.style={concept, execute at begin node=\hskip0pt},
    root concept/.append style={
        concept color=black,
        fill=white,
        line width=1ex,
        text=black,
        font=\large\bfseries
    },
    text=black,
    grow cyclic,
    level 1/.append style={level distance=4.5cm,sibling angle=90},
    level 2/.append style={level distance=3cm,sibling angle=45}
]

\node [root concept] {Рискове при Blind AI-Coding}
    child [concept color=orange!30] { node {Сигурност}
        child { node {XSS / CSRF} }
        child { node {SQL Injection} }
    }
    child [concept color=yellow!30] { node {Качество}
        child { node {Логически грешки} }
        child { node {Неоптимален код} }
    }
    child [concept color=green!30] { node {Поддръжка}
        child { node {Липса на стандарти} }
        child { node {Труден дебъг} }
    }
    child [concept color=blue!30] { node {Етика}
        child { node {Лицензи} }
        child { node {Халюцинации} }
    };
\end{tikzpicture}
\caption{Визуализация на рисковете при ``blind AI-coding''}
\label{fig:blind-ai-risks}
\end{figure}

Фигура~\ref{fig:blind-ai-risks} илюстрира опасностите от некритично приемане на AI генериран код. Основните рискове включват [30]:

\begin{itemize}
    \item \textbf{Грешки}: AI може да генерира синтактично верен, но логически грешен код
    \item \textbf{Неоптималност}: Моделите не винаги генерират най-ефективното решение
    \item \textbf{Липса на стандарти}: AI може да не следва проектните конвенции
    \item \textbf{Security}: Уязвимости като XSS, SQL injection, CSRF [31]
\end{itemize}

Особено опасен е феноменът "vibe coding" - когато разработчици слепо копират AI генериран код без да разбират как работи [32].

\subsection{Етика и професионална отговорност}
\label{subsec:ethics}

Използването на AI в разработката поражда етични въпроси [33]:

\paragraph{Халюцинации} AI моделите могат да "измислят" факти, API-та или библиотеки, които не съществуват [34]. Примери:
\begin{itemize}
    \item Измислени функции в съществуващи библиотеки
    \item Невалидни URL-и или документация
    \item Остарял код от предишни версии
\end{itemize}

\paragraph{Проверими източници} Всяка AI препоръка трябва да се валидира срещу официална документация [35]:
\begin{itemize}
    \item MDN Web Docs за JavaScript/CSS/HTML
    \item W3C спецификации за стандарти
    \item Официални GitHub repositories
    \item Stack Overflow (проверени отговори)
\end{itemize}

\paragraph{Авторство и лицензиране} Сложни въпроси за интелектуална собственост [36]:
\begin{itemize}
    \item Кой е авторът на AI генериран код?
    \item Какъв лиценз се прилага?
    \item Има ли рискове от copyright нарушения?
\end{itemize}

\subsection{Дигитален маркетинг и SEO}
\label{subsec:digital-marketing}

ChatGPT може да подпомогне дигиталния маркетинг [37]:

\begin{itemize}
    \item \textbf{Meta tags}: Генериране на оптимизирани meta description, keywords
    \item \textbf{Structured data}: Schema.org JSON-LD markup
    \item \textbf{Content optimization}: Heading hierarchy, keyword density
    \item \textbf{Alt текстове}: Accessibility и SEO описания на изображения
\end{itemize}

Пример за meta tags:
\begin{lstlisting}[language=HTML]
<meta name="description" content="Comprehensive guide to prompt 
      engineering for web development with ChatGPT">
<meta name="keywords" content="prompt engineering, ChatGPT, 
      web development, AI coding">
<meta property="og:title" content="Prompt Engineering Guide">
<meta property="og:type" content="article">
\end{lstlisting}

\subsection{Бъдещето на професиите}
\label{subsec:future}

AI променя фундаментално професията на уеб разработчика [38]:

\paragraph{Автоматизация} Рутинни задачи, които вероятно ще се автоматизират:
\begin{itemize}
    \item Boilerplate код генериране
    \item Основни CRUD операции
    \item CSS стилизация по дизайн mockups
    \item Unit test generation
\end{itemize}

\paragraph{Ефект върху обучението} Промени в образованието [39]:
\begin{itemize}
    \item Фокус върху архитектура, а не синтаксис
    \item Критично мислене и код ревю
    \item Етика и отговорност
    \item Prompt engineering като нова компетентност
\end{itemize}

\paragraph{Junior разработчици} Предизвикателства за начинаещи [40]:
\begin{itemize}
    \item Риск от "не научаване на основите"
    \item Зависимост от AI инструменти
    \item Необходимост от нови ментори стратегии
\end{itemize}

\paragraph{Разработвателни pipeline-и} Трансформация на процесите [41]:
\begin{itemize}
    \item AI-assisted code review
    \item Automated testing generation
    \item Intelligent CI/CD optimization
    \item Predictive bug detection
\end{itemize}

\section{Приложения в уеб - примери и демонстрации}
\label{sec:examples}

Тази секция представя конкретни примери и казуси за използване на prompt engineering в реални уеб проекти.

\subsection{Демонстрация на различни типове промптове}
\label{subsec:prompt-demos}

Разгледахме вече четири типа промптове в предишните секции:
\begin{itemize}
    \item Код~\ref{code:basic-prompt}: Basic prompt
    \item Код~\ref{code:advanced-prompt}: Advanced prompt
    \item Код~\ref{code:cot}: Chain-of-thought
    \item Код~\ref{code:few-shot}: Few-shot learning
\end{itemize}

\subsection{Генериран код с добра структура}
\label{subsec:good-code}

Демонстрирахме примери на добре структуриран код:
\begin{itemize}
    \item Код~\ref{code:html-template}: Семантичен HTML5
    \item Код~\ref{code:css-grid}: Модерен CSS Grid
    \item Код~\ref{code:js-dom}: JavaScript с security best practices
    \item Код~\ref{code:i18n}: i18n архитектура
\end{itemize}

\subsection{Мини-казуси: кога AI помага и кога пречи}
\label{subsec:case-studies}

\paragraph{Случай 1: Успешна помощ} 
Разработчик използва ChatGPT за генериране на responsive navigation menu. AI генерира чист код с accessibility features. След малки корекции, кодът е production-ready [42].

\paragraph{Случай 2: Проблематично използване}
Junior разработчик поиска от AI да генерира authentication система. AI създаде код без proper password hashing и с SQL injection уязвимости. Само след security audit проблемите са открити [43].

\paragraph{Случай 3: Оптимален workflow}
Senior team използва AI за boilerplate, но всички резултати преминават през:
\begin{enumerate}
    \item Code review
    \item Automated testing
    \item Security scanning
    \item Performance profiling
\end{enumerate}

Резултат: 35\% по-бърза разработка при запазване на качеството [44].

\subsection{Как да валидираме резултата}
\label{subsec:how-validate}

Систематичен подход за валидация [45]:

\begin{table}[h]
\centering
\caption{Checklist за валидация на AI генериран код}
\label{tab:validation-checklist}
\begin{tabular}{@{}p{0.3\textwidth}p{0.6\textwidth}@{}}
\toprule
\textbf{Категория} & \textbf{Проверки} \\ \midrule
Функционалност & Работи ли кодът както се очаква? Unit tests pass? \\
Performance & Няма ли memory leaks? Оптимизиран ли е? \\
Security & OWASP Top 10 проверки, input validation \\
Accessibility & WCAG 2.1 compliance, keyboard navigation \\
Standards & Съответствие с проектни конвенции \\
Documentation & Коментари, JSDoc, README updates \\
Cross-browser & Тестване в Chrome, Firefox, Safari, Edge \\
Responsive & Mobile, tablet, desktop layouts \\ \bottomrule
\end{tabular}
\end{table}

\section{Рискове и ограничения}
\label{sec:risks}

Въпреки предимствата, използването на AI в уеб разработката носи значителни рискове, които изискват внимание и митигация [46].

\subsection{Етически предизвикателства}
\label{subsec:ethical-challenges}

\paragraph{Прозрачност} Трябва ли да се декларира, че код е AI-генериран? [47]
\begin{itemize}
    \item В open-source проекти
    \item В комерсиални продукти
    \item При code contributions
\end{itemize}

\paragraph{Bias и дискриминация} AI моделите могат да възпроизведат предразсъдъци от тренировъчните данни [48]:
\begin{itemize}
    \item Стереотипни UI/UX решения
    \item Ограничена accessibility
    \item Cultural bias в съдържанието
\end{itemize}

\subsection{Надеждност и халюцинации}
\label{subsec:reliability}

Статистика показва, че GPT-4 може да генерира невалиден код в 12-18\% от случаите [49]. Типични проблеми:
\begin{itemize}
    \item Измислени API endpoints
    \item Deprecated функции
    \item Неправилни параметри
    \item Логически грешки
\end{itemize}

\subsection{Сигурност}
\label{subsec:security}

AI генерираният код може да съдържа уязвимости [50]:

\begin{itemize}
    \item \textbf{XSS}: Недостатъчна input санитизация
    \item \textbf{SQL Injection}: Директно вмъкване на user input
    \item \textbf{CSRF}: Липса на токени
    \item \textbf{Hardcoded secrets}: API keys в кода
    \item \textbf{Insecure dependencies}: Остарели библиотеки
\end{itemize}

\subsection{Неправилни цитати}
\label{subsec:citation-issues}

AI може да генерира невалидни или измислени цитати [51]:
\begin{itemize}
    \item Несъществуващи статии
    \item Грешни автори или дати
    \item Невалидни URLs
    \item Объркани източници
\end{itemize}

\textbf{Решение}: Винаги проверявайте цитатите в оригиналния източник!

\subsection{Критични контролни точки}
\label{subsec:checkpoints}

Задължителни проверки при използване на AI код [52]:

\begin{enumerate}
    \item \textbf{Pre-commit}: Static analysis, linting
    \item \textbf{Code review}: Human inspection
    \item \textbf{Testing}: Unit, integration, e2e tests
    \item \textbf{Security scan}: SAST/DAST tools
    \item \textbf{Performance}: Lighthouse, profiling
    \item \textbf{Accessibility}: axe, WAVE audits
    \item \textbf{Production monitoring}: Error tracking, analytics
\end{enumerate}

\section{Заключение}
\label{sec:conclusion}

Prompt engineering представлява мощен инструмент за модерната уеб разработка, който може значително да ускори и подобри процеса на създаване на софтуер [53]. Въпреки това, технологията изисква отговорен и критичен подход.

\subsection{Авторско мнение}
\label{subsec:author-opinion}

Въз основа на изследването в този реферат, считам, че ChatGPT и prompt engineering \textbf{трябва да се използват}, но при следните условия:

\paragraph{Подходящи сценарии:}
\begin{itemize}
    \item Генериране на boilerplate код
    \item Прототипиране и експериментиране
    \item Документация и коментари
    \item Refactoring suggestions
    \item Learning и образование (с ограничения)
\end{itemize}

\paragraph{Неподходящи сценарии:}
\begin{itemize}
    \item Critical security код без review
    \item Production код без тестове
    \item Единственият източник на обучение за начинаещи
    \item Сложна бизнес логика без domain expertise
\end{itemize}

\subsection{Отговорни практики}
\label{subsec:responsible}

Препоръки за отговорно използване [54]:

\begin{enumerate}
    \item \textbf{Transparency}: Документирайте AI употреба
    \item \textbf{Validation}: Винаги проверявайте резултатите
    \item \textbf{Education}: Обучавайте се на основите преди да използвате AI
    \item \textbf{Testing}: Comprehensive test coverage
    \item \textbf{Security}: Regular security audits
    \item \textbf{Ethics}: Съобразявайте се с етичните стандарти
    \item \textbf{Continuous learning}: AI еволюира, адаптирайте се
\end{enumerate}

\subsection{Бъдещи перспективи}
\label{subsec:future-perspectives}

В следващите 5-10 години очаквам [55]:

\begin{itemize}
    \item \textbf{Специализирани модели}: AI обучени конкретно за web frameworks
    \item \textbf{IDE интеграция}: Seamless AI assistance в dev tools
    \item \textbf{Автоматизация}: End-to-end код генериране от design
    \item \textbf{Нови роли}: AI prompt engineers, AI code reviewers
    \item \textbf{Стандарти}: Industry guidelines за AI използване
\end{itemize}

\subsection{Финални думи}
\label{subsec:final-words}

Prompt engineering не е магическо решение, а инструмент, който изисква умение, критично мислене и професионална отговорност. Успешните разработчици ще бъдат тези, които комбинират фундаментални знания с ефективно използване на AI технологии [56].

Бъдещето на уеб разработката е в симбиозата между човешка креативност и AI възможности - не в замяната на едното с другото.

\newpage

\section*{Цитирана литература}
\addcontentsline{toc}{section}{Цитирана литература}

\begin{enumerate}
\item OpenAI, "GPT-4 Technical Report", published March 2023, OpenAI Research, [\url{https://arxiv.org/abs/2303.08774}], last visited: 2025-12-01.

\item Patel, N., Raman, K., "The Impact of ChatGPT on Software Development", published April 2023, Communications of the ACM, vol. 66, no. 4, [\url{https://cacm.acm.org/magazines/2023/4/271234}], last visited: 2025-12-01.

\item White, J., et al., "A Prompt Pattern Catalog to Enhance Prompt Engineering with ChatGPT", published February 2023, arXiv preprint, [\url{https://arxiv.org/abs/2302.11382}], last visited: 2025-12-01.

\item Liu, P., et al., "Pre-train, Prompt, and Predict: A Systematic Survey of Prompting Methods in Natural Language Processing", published 2023, ACM Computing Surveys, vol. 55, no. 9, [\url{https://dl.acm.org/doi/10.1145/3560815}], last visited: 2025-12-01.

\item McKinsey Digital, "The State of AI in 2023: Generative AI's Breakout Year", published August 2023, McKinsey \& Company, [\url{https://www.mckinsey.com/capabilities/quantumblack/our-insights/the-state-of-ai-in-2023}], last visited: 2025-12-02.

\item GitHub, "The Impact of AI on Developer Productivity", published September 2023, GitHub Blog, [\url{https://github.blog/2023-09-27-the-impact-of-ai-on-developer-productivity/}], last visited: 2025-12-02.

\item Ziegler, A., et al., "Productivity Assessment of Neural Code Completion", published May 2022, Microsoft Research, [\url{https://arxiv.org/abs/2205.06537}], last visited: 2025-12-02.

\item Reynolds, L., McDonell, K., "Prompt Programming for Large Language Models", published February 2021, arXiv preprint, [\url{https://arxiv.org/abs/2102.07350}], last visited: 2025-12-02.

\item Brown, T., et al., "Language Models are Few-Shot Learners", published 2020, NeurIPS 2020, [\url{https://papers.nips.cc/paper/2020/hash/1457c0d6bfcb4967418bfb8ac142f64a-Abstract.html}], last visited: 2025-12-02.

\item OpenAI, "Best Practices for Prompt Engineering", OpenAI Documentation, [\url{https://platform.openai.com/docs/guides/prompt-engineering}], last visited: 2025-12-02.

\item Vaswani, A., et al., "Attention Is All You Need", published 2017, NeurIPS 2017, [\url{https://arxiv.org/abs/1706.03762}], last visited: 2025-12-02.

\item Zhou, Y., et al., "Large Language Models Are Human-Level Prompt Engineers", published 2023, ICLR 2023, [\url{https://arxiv.org/abs/2211.01910}], last visited: 2025-12-02.

\item Shanahan, M., et al., "Role-Play with Large Language Models", published October 2023, Nature Machine Intelligence, [\url{https://www.nature.com/articles/s42256-023-00754-x}], last visited: 2025-12-02.

\item Kojima, T., et al., "Large Language Models are Zero-Shot Reasoners", published 2022, NeurIPS 2022, [\url{https://arxiv.org/abs/2205.11916}], last visited: 2025-12-02.

\item Dettmers, T., et al., "GPT3.int8(): 8-bit Matrix Multiplication for Transformers at Scale", published 2022, NeurIPS 2022, [\url{https://arxiv.org/abs/2208.07339}], last visited: 2025-12-02.

\item Madaan, A., et al., "Self-Refine: Iterative Refinement with Self-Feedback", published March 2023, arXiv preprint, [\url{https://arxiv.org/abs/2303.17651}], last visited: 2025-12-02.

\item Dong, Q., et al., "A Survey on In-context Learning", published December 2022, arXiv preprint, [\url{https://arxiv.org/abs/2301.00234}], last visited: 2025-12-02.

\item Wei, J., et al., "Chain-of-Thought Prompting Elicits Reasoning in Large Language Models", published 2022, NeurIPS 2022, [\url{https://arxiv.org/abs/2201.11903}], last visited: 2025-12-02.

\item Mitchell, M., "Why AI is Harder Than We Think", published April 2021, arXiv preprint, [\url{https://arxiv.org/abs/2104.12871}], last visited: 2025-12-02.

\item Wang, X., et al., "Self-Consistency Improves Chain of Thought Reasoning in Language Models", published 2023, ICLR 2023, [\url{https://arxiv.org/abs/2203.11171}], last visited: 2025-12-02.

\item Schick, T., et al., "Toolformer: Language Models Can Teach Themselves to Use Tools", published February 2023, arXiv preprint, [\url{https://arxiv.org/abs/2302.04761}], last visited: 2025-12-02.

\item Chen, M., et al., "Evaluating Large Language Models Trained on Code", published July 2021, arXiv preprint, [\url{https://arxiv.org/abs/2107.03374}], last visited: 2025-12-02.

\item Jiang, E., et al., "Promptmaker: Prompt-based Prototyping with Large Language Models", published 2022, CHI 2022, [\url{https://dl.acm.org/doi/10.1145/3491101.3519729}], last visited: 2025-12-02.

\item Dakhel, A., et al., "GitHub Copilot AI pair programmer: Asset or Liability?", published September 2023, Journal of Systems and Software, vol. 203, [\url{https://www.sciencedirect.com/science/article/pii/S0164121223001292}], last visited: 2025-12-02.

\item MDN Web Docs, "CSS Grid Layout", Mozilla Developer Network, [\url{https://developer.mozilla.org/en-US/docs/Web/CSS/CSS_Grid_Layout}], last visited: 2025-12-02.

\item MDN Web Docs, "Manipulating Documents", Mozilla Developer Network, [\url{https://developer.mozilla.org/en-US/docs/Learn/JavaScript/Client-side_web_APIs/Manipulating_documents}], last visited: 2025-12-02.

\item Pressman, R., Maxim, B., "Software Engineering: A Practitioner's Approach", published 2020, 9th Edition, McGraw-Hill Education.

\item W3C, "Internationalization Best Practices", W3C Working Draft, [\url{https://www.w3.org/TR/i18n-html-tech-lang/}], last visited: 2025-12-02.

\item Humble, J., Farley, D., "Continuous Delivery: Reliable Software Releases through Build, Test, and Deployment Automation", published 2010, Addison-Wesley.

\item Perry, N., et al., "Do Users Write More Insecure Code with AI Assistants?", published August 2023, arXiv preprint, [\url{https://arxiv.org/abs/2211.03622}], last visited: 2025-12-02.

\item OWASP, "OWASP Top Ten 2021", OWASP Foundation, [\url{https://owasp.org/www-project-top-ten/}], last visited: 2025-12-02.

\item Barke, S., et al., "Grounded Copilot: How Programmers Interact with Code-Generating Models", published 2023, OOPSLA 2023, [\url{https://arxiv.org/abs/2206.15000}], last visited: 2025-12-02.

\item Bender, E., et al., "On the Dangers of Stochastic Parrots: Can Language Models Be Too Big?", published 2021, FAccT 2021, [\url{https://dl.acm.org/doi/10.1145/3442188.3445922}], last visited: 2025-12-02.

\item Ji, Z., et al., "Survey of Hallucination in Natural Language Generation", published March 2023, ACM Computing Surveys, vol. 55, no. 12, [\url{https://arxiv.org/abs/2202.03629}], last visited: 2025-12-02.

\item MDN Web Docs, "HTML: HyperText Markup Language", Mozilla Developer Network, [\url{https://developer.mozilla.org/en-US/docs/Web/HTML}], last visited: 2025-12-02.

\item Samuelson, P., "Allocating Ownership Rights in Computer-Generated Works", published 1986, University of Pittsburgh Law Review, vol. 47.

\item Search Engine Journal, "AI for SEO: Complete Guide", published October 2023, Search Engine Journal, [\url{https://www.searchenginejournal.com/ai-seo/}], last visited: 2025-12-02.

\item World Economic Forum, "The Future of Jobs Report 2023", published April 2023, WEF, [\url{https://www.weforum.org/reports/the-future-of-jobs-report-2023}], last visited: 2025-12-02.

\item ACM, "Computing Curricula 2020", published December 2020, ACM/IEEE Computer Society, [\url{https://www.acm.org/education/curricula-recommendations}], last visited: 2025-12-02.

\item Prather, J., et al., "The Robots Are Here: Navigating the Generative AI Revolution in Computing Education", published June 2023, ACM Inroads, vol. 14, no. 2.

\item Forsgren, N., et al., "Accelerate: The Science of Lean Software and DevOps", published 2018, IT Revolution Press.

\item Stack Overflow, "2023 Developer Survey", published May 2023, Stack Overflow, [\url{https://survey.stackoverflow.co/2023/}], last visited: 2025-12-02.

\item Nguyen, N., Nadi, S., "An Empirical Evaluation of GitHub Copilot's Code Suggestions", published 2022, MSR 2022, [\url{https://arxiv.org/abs/2205.06537}], last visited: 2025-12-02.

\item Ross, S., et al., "Programmer Experiences with GitHub Copilot", published August 2023, arXiv preprint, [\url{https://arxiv.org/abs/2303.06104}], last visited: 2025-12-02.

\item Google Developers, "Web Fundamentals - Performance", Google, [\url{https://developers.google.com/web/fundamentals/performance}], last visited: 2025-12-02.

\item Thorp, H., "ChatGPT is fun, but not an author", published January 2023, Science, vol. 379, no. 6630.

\item Kalluri, P., "Don't ask if artificial intelligence is good or fair, ask how it shifts power", published July 2020, Nature, vol. 583.

\item Buolamwini, J., Gebru, T., "Gender Shades: Intersectional Accuracy Disparities in Commercial Gender Classification", published 2018, FAT 2018.

\item Borji, A., "A Categorical Archive of ChatGPT Failures", published February 2023, arXiv preprint, [\url{https://arxiv.org/abs/2302.03494}], last visited: 2025-12-02.

\item Pearce, H., et al., "Asleep at the Keyboard? Assessing the Security of GitHub Copilot's Code Contributions", published May 2022, IEEE Symposium on Security and Privacy.

\item Alkaissi, H., McFarlane, S., "Artificial Hallucinations in ChatGPT: Implications in Scientific Writing", published February 2023, Cureus.

\item Bass, L., et al., "Software Architecture in Practice", published 2021, 4th Edition, Addison-Wesley.

\item Radford, A., et al., "Language Models are Unsupervised Multitask Learners", published 2019, OpenAI Blog, [\url{https://openai.com/research/better-language-models}], last visited: 2025-12-02.

\item IEEE, "IEEE Code of Ethics", IEEE, [\url{https://www.ieee.org/about/corporate/governance/p7-8.html}], last visited: 2025-12-02.

\item Brynjolfsson, E., et al., "Generative AI at Work", published April 2023, NBER Working Paper, [\url{https://www.nber.org/papers/w31161}], last visited: 2025-12-02.

\item Hutson, M., "Could AI help you to write better code?", published April 2023, Nature, vol. 616, [\url{https://www.nature.com/articles/d41586-023-00983-1}], last visited: 2025-12-02.

\end{enumerate}

\newpage

\section*{Приложения}
\addcontentsline{toc}{section}{Приложения}

\subsection*{Приложение А: Кратък речник на термини}
\begin{itemize}
    \item \textbf{LLM} - Large Language Model, голям езиков модел
    \item \textbf{Prompt} - Инструкция или въпрос към AI модел
    \item \textbf{Token} - Базова единица текст за обработка от модела
    \item \textbf{Few-shot learning} - Обучение с малко примери
    \item \textbf{Chain-of-thought} - Стъпка-по-стъпка разсъждение
    \item \textbf{Hallucination} - "Измисляне" на факти от AI
    \item \textbf{i18n} - Internationalization, интернационализация
    \item \textbf{OWASP} - Open Web Application Security Project
\end{itemize}

\subsection*{Приложение Б: Полезни ресурси}
\begin{itemize}
    \item MDN Web Docs: \url{https://developer.mozilla.org/}
    \item W3C Standards: \url{https://www.w3.org/standards/}
    \item OpenAI Documentation: \url{https://platform.openai.com/docs}
    \item OWASP Top 10: \url{https://owasp.org/www-project-top-ten/}
    \item CSS-Tricks: \url{https://css-tricks.com/}
\end{itemize}

\subsection*{Приложение В: Примерни тестови въпроси}
\label{app:questions}

\begin{enumerate}
    \item \textbf{Какво представлява "Prompt Engineering"?}
    \begin{enumerate}
        \item Процес на хардуерна оптимизация на сървъри
        \item \textbf{Процес на проектиране и оптимизация на инструкции към AI модели}
        \item Метод за компилиране на JavaScript код
        \item Техника за управление на бази данни
    \end{enumerate}

    \item \textbf{Коя от следните техники включва предоставяне на примери в промпта?}
    \begin{enumerate}
        \item Zero-shot learning
        \item \textbf{Few-shot learning}
        \item Reinforcement learning
        \item Unsupervised learning
    \end{enumerate}

    \item \textbf{Какво е "Chain-of-Thought" (CoT) промптване?}
    \begin{enumerate}
        \item Свързване на няколко AI модела в мрежа
        \item \textbf{Насърчаване на модела да обяснява стъпките в разсъжденията си}
        \item Използване на блокчейн технологии в AI
        \item Генериране на безкраен цикъл от промптове
    \end{enumerate}

    \item \textbf{Какъв е основният риск при "Blind AI-Coding"?}
    \begin{enumerate}
        \item AI моделът ще спре да работи
        \item \textbf{Внедряване на код с грешки или уязвимости без проверка}
        \item Кодът ще бъде твърде оптимизиран
        \item Загуба на интернет връзка
    \end{enumerate}

    \item \textbf{Какво представляват "халюцинациите" при LLM?}
    \begin{enumerate}
        \item Визуални ефекти в интерфейса
        \item \textbf{Генериране на невярна или измислена информация, представена като факт}
        \item Прегряване на графичните процесори
        \item Автоматично изтриване на данни
    \end{enumerate}

    \item \textbf{Кой инструмент е подходящ за валидация на AI генериран JavaScript код?}
    \begin{enumerate}
        \item Photoshop
        \item \textbf{ESLint}
        \item Microsoft Word
        \item SQLite
    \end{enumerate}

    \item \textbf{Как AI може да помогне при i18n (интернационализация)?}
    \begin{enumerate}
        \item Чрез автоматично превеждане на потребителите
        \item \textbf{Чрез генериране на JSON структури с преводи за различни езици}
        \item Чрез забрана на достъпа от чужбина
        \item Чрез промяна на часовата зона на сървъра
    \end{enumerate}

    \item \textbf{Какво е "Role Prompting"?}
    \begin{enumerate}
        \item Задаване на роля на потребителя в системата
        \item \textbf{Инструкция към AI модела да действа като експерт в дадена област}
        \item Ролева игра между разработчици
        \item Административен достъп до модела
    \end{enumerate}

    \item \textbf{Защо е важно да се проверяват цитатите, генерирани от AI?}
    \begin{enumerate}
        \item Защото AI не може да пише на английски
        \item \textbf{Защото AI често генерира несъществуващи или грешни източници}
        \item Защото цитатите заемат много място
        \item Няма нужда да се проверяват
    \end{enumerate}

    \item \textbf{Коя е добра практика при използване на AI в CI/CD pipeline?}
    \begin{enumerate}
        \item Автоматично деплойване на всичко генерирано
        \item \textbf{Включване на автоматизирани тестове и security сканиране}
        \item Изключване на всички валидации
        \item Премахване на човешкия контрол
    \end{enumerate}
\end{enumerate}

\subsection*{Приложение Г: Изисквания и задание}
\label{app:requirements}

\textbf{Тема №187: Prompt engineering и Web разработка с ChatGPT}

\textbf{Цел на проекта:} Разработване на интерактивен, семантичен и стилизиран информационен сайт по зададената тема, използвайки съвременни уеб технологии.

\textbf{Основни изисквания (25 издание):}
\begin{itemize}
    \item Обем: ~15 страници (вкл. фигури, код, таблици).
    \item Език: Български.
    \item Ресурси: Само на английски език, коректно цитирани.
    \item Съдържание: Достоверна и проверена информация.
    \item Технологии: HTML5, CSS3, JavaScript (за уеб версията).
    \item Структура: Увод, Изложение (с примери), Заключение, Библиография.
\end{itemize}

\textbf{Специфични изисквания за темата:}
\begin{itemize}
    \item Дефиниране на Prompt Engineering.
    \item Анализ на техники (Few-shot, CoT, Role-playing).
    \item Приложение в уеб разработката (HTML/CSS/JS generation).
    \item Етични съображения и рискове.
    \item Бъдеще на професията.
\end{itemize}

\end{document}
